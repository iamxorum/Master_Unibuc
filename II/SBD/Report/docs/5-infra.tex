\chapter{Controlul Accesului}

Pentru referatul acesta, o să abordez 3 tipuri de access control (menționate și la cursul SBD):

\begin{itemize}
    \item \textbf{RBAC}: Role Based Access Control;
    \item \textbf{MAC}: Mandatory Access Control;
    \item \textbf{DAC}: Discretionary Access Control;
\end{itemize}

\section{Role Based Access Control}

Acest tip de access control bazează accessul prin roluri. În loc de a configura pentru fiecare personal\_medical ce are voie să facă, să vadă, să modifice, să steargă (CRUD), stabilim asta pe baza de rol așa că daca e nevoie să se steargă/adauge/modifce o regulă de access, este mai ușor de intretinut/mentinut pentru ca se face doar odată schimbarea.

\begin{figure}[h]
    \centering 
    \includegraphics[width=0.4\textwidth]{../assets/RBAC/Access_RBAC.png}
    \caption{Configurare RBAC}
    \label{fig:configurare_rbac}
\end{figure}

După ce am creat rolurile/userii pentru acces, putem testa (prin linie de comandă).

Se poate observa că medicul chiar are acces la orice (poate să adauge/șteargă pacienți noi spre exemplu), când asistentul vede totul dar nu are drept de modificare/ștergere/adaugare a datelor. În final, rezidentul are acces doar la fișele medicale.

\begin{figure}[h]
    \centering
    \begin{minipage}{0.45\textwidth}
        \centering
        \includegraphics[width=\textwidth]{../assets/RBAC/Screen1.png}
        \caption{Rol Medic}
        \label{fig:rol_medic}
    \end{minipage}
    \hfill
    \begin{minipage}{0.45\textwidth}
        \centering
        \includegraphics[width=\textwidth]{../assets/RBAC/Screen2.png}
        \caption{Rol Asistent}
        \label{fig:rol_asistent}
    \end{minipage}
\end{figure}

\begin{figure}[h]
    \centering
    \begin{minipage}{0.45\textwidth}
        \centering
        \includegraphics[width=\textwidth]{../assets/RBAC/Screen3.png}
        \caption{Rol Rezident}
        \label{fig:rol_rezident}
    \end{minipage}
    \hfill
    \begin{minipage}{0.45\textwidth}
        \centering
        \includegraphics[width=\textwidth]{../assets/RBAC/Screen4.png}
        \caption{Verificare Roluri}
        \label{fig:verificare_roluri}
    \end{minipage}
\end{figure}

\newpage
\section{Discretionary Access Control}

Acest tip de access control permite proprietarului unui obiect (tabel, înregistrare) să decidă direct cine are acces. Permisiunile sunt acordate individual fiecărui utilizator, oferind control granular asupra accesului la date.

Cum se poate observa, după ce medicul a primit rolul de SELECT access pe "fisa\_medicala" cu opțiune de a da mai departe, userul "elena.constantinescu" (asistent) a primit permisiunea de SELECT pe același tabel și se poate observa că a primit accesul.

\begin{figure}[h]
    \centering 
    \includegraphics[width=0.4\textwidth]{../assets/DAC/Screen1.png}
    \caption{Testare DAC}
    \label{fig:testare_dac}
\end{figure}

\section{Mandatory Access Control}

Acest tip de access control este impus de sistem pe baza unor etichete de securitate și reguli stricte (în cazul acesta "grad\_acreditare" pentru personal și "nivel\_clasificare" pentru fișele medicale). Accesul este determinat de nivelul de clasificare al datelor și al utilizatorului, fără posibilitatea modificării de către utilizatori (prin intermediul RLS - Row Level Security).

După ce s-a rulat acces-ul pentru RLS, se poate vedea cum fiecare user poate să vadă doar înregistrările care au "nivel\_clasificare" mai mic sau egal cu propriul lor "grad\_acreditare".

\begin{figure}[h]
    \centering 
    \includegraphics[width=0.4\textwidth]{../assets/MAC/Screen1.png}
    \caption{Testare MAC}
    \label{fig:testare_mac}
\end{figure}