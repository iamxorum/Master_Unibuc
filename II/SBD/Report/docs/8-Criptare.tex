\chapter{Criptarea Datelor}

Criptarea datelor este necesară pentru protejarea informațiilor sensibile împotriva accesului neautorizat.

\section{Full Disk Encryption - La nivel de disk}
Una dintre cele mai bune metode de protejare a datelor este criptarea completă a discului sau a partiției. Această tehnică asigură protecția atât a fiecărui fișier, cât și a stocării temporare care poate conține părți ale acestora. Nu trebuie să te îngrijorezi să alegi ce fișier vrei să protejezi, deoarece criptarea completă a discului protejează toate fișierele. \cite{edb_security_2020}

Avem diverse opțiuni (depinde de OS-ul pe care îl folosim):
Implementarea folosește un tabel "audit\_log" care stochează:
\begin{itemize}
    \item \textbf{MacOS - FileVault};
    \item \textbf{Windows - BitLocker};
    \item \textbf{Linux - LUKS (Linux Unified Key Setup)}.
\end{itemize}

\begin{figure}[h]
    \centering
    \begin{minipage}{0.45\textwidth}
        \centering
        \includegraphics[width=\textwidth]{../assets/encryption/FileVault.png}
        \caption{FileVault}
        \label{fig:filvault}
    \end{minipage}
    \hfill
    \begin{minipage}{0.45\textwidth}
        \centering
        \includegraphics[width=\textwidth]{../assets/encryption/BitLocker.png}
        \caption{BitLocker}
        \label{fig:bitlocker}
    \end{minipage}
\end{figure}

\section{File System Encryption - La nivel de os}

Criptarea la nivel de sistem de fișiere, denumită și criptare de fișiere sau directoare, este atunci când sistemul de fișiere însuși criptează fișierele sau directoarele. \cite{edb_security_2020}

\section{Criptarea la nivel de coloană}

Pentru criptare la nivel de coloană, se poate folosi extensia "pgcrypto". Aceasta permite criptarea selectivă a coloanelor cu date sensibile, cum ar fi numerele CNP și cardurile de credit. \cite{edb_security_2020}

\begin{figure}[h]
    \centering 
    \includegraphics[width=0.5\textwidth]{../assets/encryption/Screenshot 2025-12-31 at 19.59.16.png}
    \caption{encryption.sql}
    \label{fig:encryption.sql}
\end{figure}

În cazul tabelelor pe care le avem, am creat o entitate nouă pentru a arăta diferența între a citi având cheia de decriptare și fără cheie.

După ce am creat uneltele necesare (verificați "encryption.sql" din imagine), mai jos se poate observa diferența când, după ce s-a introdus pacientul cu criptarea CNP-ului, se face citirea fără sau cu cheie.

\begin{figure}[h]
    \centering
    \begin{minipage}{0.45\textwidth}
        \centering
        \includegraphics[width=\textwidth]{../assets/encryption/Select_cu_cheie.png}
        \caption{Cu cheie}
        \label{fig:cu_cheie}
    \end{minipage}
    \hfill
    \begin{minipage}{0.45\textwidth}
        \centering
        \includegraphics[width=\textwidth]{../assets/encryption/Select_fara_cheie.png}
        \caption{Fără cheie}
        \label{fig:fara_cheie}
    \end{minipage}
\end{figure}

Pentru tabelul curent "pacient", am criptat coloana prin comanda următoare:

\begin{lstlisting}[caption={Migrarea coloanei CNP la format criptat (BYTEA)}, label={lst:encrypt_update}]
ALTER TABLE pacient
ALTER COLUMN cnp TYPE BYTEA
USING encrypt_cnp(cnp, 'portocal12');
\end{lstlisting}

Jos se poate observa coloana "cnp" din tabelul "pacient" criptat după rularea comenzii de mai sus:

\begin{figure}[h]
    \centering 
    \includegraphics[width=0.5\textwidth]{../assets/encryption/Pacient_criptat.png}
    \caption{CNP Criptat}
    \label{fig:cnp_criptat}
\end{figure}

\section{Criptare la nivel de rețea}

Criptarea la nivel de rețea protejează datele în tranzit între client și server. În PostgreSQL, aceasta se realizează prin SSL/TLS. Am menționat anterior opțiunea "hostssl" în "pg\_hba.conf" care necesită conexiuni SSL.

\section{Criptarea paroleor}
PostgreSQL folosește deja criptare pentru stocarea parolelor. Am menționat anterior folosirea "scram-sha-256" în loc de "md5" pentru autentificare.