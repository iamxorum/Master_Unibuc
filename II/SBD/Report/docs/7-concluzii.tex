\chapter{Prevenirea SQL Injection}

SQL Injection este printre cele mai frecvente atacuri asupra bazelor de date, în care un atacator injectează cod SQL fals în interogări prin concatenarea string-urilor. Soluția implică interogările parametrizate pregătite, care separă codul SQL de datele utilizatorului.

În PostgreSQL, se pot folosi prepared statements cu "PREPARE" și "EXECUTE", sau funcții PL/pgSQL care acceptă parametri. Acestea previne injectarea de cod SQL deoarece parametrii sunt tratați ca date, nu ca parte a comenzii SQL.

\begin{figure}[h]
    \centering
    \begin{minipage}{0.45\textwidth}
        \centering
        \includegraphics[width=\textwidth]{../assets/injection/Screen1.png}
        \caption{Testare Injection}
        \label{fig:testare_injection}
    \end{minipage}
    \hfill
    \begin{minipage}{0.45\textwidth}
        \centering
        \includegraphics[width=\textwidth]{../assets/injection/Screen2.png}
        \caption{Verificare Injection}
        \label{fig:verificare_injection}
    \end{minipage}
\end{figure}