\chapter{Schema bazei de date}

Pentru referatul acesta (care va fi implementat în continuare pentru aplicația de evaluare SBD) o să mă folosesc de un DB Schema simplu compus de 3 entități:

\begin{itemize}
    \item \textbf{pacient};
    \item \textbf{personal\_medical};
    \item \textbf{fisa\_medicala};
\end{itemize}

Relațiile sunt următoarele:

\begin{itemize}
    \item \textbf{Pacient (1) - (N) Fisa\_medicala};
    \item \textbf{Personal\_medical (1) - (N) Fisa\_medicala};
\end{itemize}

Jos puteți observa rularea creării a schemei:

\begin{figure}[h]
    \centering 
    \includegraphics[width=0.5\textwidth]{../assets/Schema_creation.png}
    \caption{Creare schema}
    \label{fig:creare_schema}
\end{figure}