\begin{abstractpage}

\begin{abstract}{romanian}
În acest referat se efectuează o analiză detaliată a mecanismelor de securitate disponibile în ediția "Community a PostgreSQL", demonstrând cum pot fi utilizate pe o bază de date medicală simplă formată din trei entități principale: "pacient", "personal\_medical" și "fișă\_medicală". Lucrarea abordează mai multe straturi de securitate, inclusiv securitatea rețelei prin utilizarea metodelor de autentificare sigure "(scram-sha-256)" și configurarea fișierului "pg\_hba.conf" pentru controlul accesului bazat pe IP. Cele paradigme fundamentale ale controlului accesului sunt examinate: Gestionarea accesului bazat pe rol "(RBAC)", controlul accesului discreționar "(DAC)" și controlul accesului obligatoriu "(MAC)" care sunt implementate prin siguranța nivelului rândului "(RLS)". Referatul prezintă modul de implementare a unui sistem de auditare care utilizează trigger-uri PostgreSQL pentru înregistrarea tuturor operațiunilor legate de date sensibile, precum și metode de prevenire a atacurilor de tip "SQL Injection" prin utilizarea declarațiilor pregătite. În ceea ce privește criptarea, există o serie de metode diferite. Acestea includ criptarea la nivel de disc complet folosind programe precum "FileVault", "BitLocker" și "LUKS"; criptarea la nivel de coloană folosind extensia "pgcrypto"; criptarea în tranzit prin "SSL/TLS". În cele din urmă, se arată cum se utilizează vizualizările "(Views)" pentru a îmbunătăți securitatea datelor prin ascunderea structurii tabelelor și limitarea accesului la date.
Totul este și visibl public in repsoitory-ul meu de Github la urmatorul Link: https://github.com/iamxorum/Master\_Unibuc/tree/main/SBD/Report
\end{abstract}

\end{abstractpage}