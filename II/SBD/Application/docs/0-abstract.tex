\begin{abstractpage}

\begin{abstract}{romanian}
Lucrarea prezintă un model de bază de date medicală securizată, CareConnect, proiectat în Oracle Database și organizat în jurul entităților pacient, personal medical, fișă medicală și departament. Securitatea datelor sensibile (CNP) este asigurată prin criptare cu pachetul \texttt{DBMS\_CRYPTO} folosind algoritmul AES-256 în modul CBC și prin gestionarea cheilor într-un tabel dedicat, cu mecanisme de rotire și auditare. Controlul accesului este implementat prin Role-Based Access Control (RBAC), Virtual Private Database (VPD) cu securitate la nivel de rând și view-uri cu mascare de date, completate de auditare standard, trigger-i de auditare și Fine-Grained Auditing (FGA). Toate aceste mecanisme sunt integrate și demonstrate practic într-o aplicație CLI Python.
\end{abstract}

\end{abstractpage}
        