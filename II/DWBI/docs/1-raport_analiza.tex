\chapter{Descrierea Modelului Ales și a Obiectivelor Aplicației}
\label{sec:descriere}

\section{Modelul de date ales}

TickLy este un sistem de management al tichetelor de suport care permite gestionarea a cererilor de asistență de la clienți. Modelul de date ales este un \textbf{model entitate-relație} extins care suportă atât clienți persoane fizice, cât și juridice, cu un sistem flexibil de categorisire și urmărire a tichetelor.

Sistemul este organizat în două straturi principale:
\begin{itemize}
    \item \textbf{Baza de date OLTP} - pentru operațiunile transacționale zilnice (creare tichete, comentarii, asignări agenți)
    \item \textbf{Data Warehouse} - pentru analize și raportare (model stea pentru analize multidimensionale)
\end{itemize}

\section{Obiectivele Aplicației}

Obiectivele principale ale aplicației TickLy sunt:

\begin{enumerate}
    \item \textbf{Gestionarea a tichetelor de suport} - urmărirea completă a ciclului de viață al tichetelor de la creare până la rezolvare
    \item \textbf{Analiza performanțelor agenților} - monitorizarea timpului de rezolvare, numărul de tichete rezolvate, rating-urile primite
    \item \textbf{Analiza satisfacției clienților} - urmărirea feedback-ului și identificarea tendințelor
    \item \textbf{Optimizare} - analiza distribuției tichetelor pe departamente și agenți
    \item \textbf{Identificarea problemelor recurente} - analiza categoriilor și subiectelor pentru a identifica problemele cele mai frecvente
    \item \textbf{Raportare} - generarea de rapoarte pentru management
\end{enumerate}

\chapter{Diagramele Bazei de Date OLTP}
\label{sec:diagrame_oltp}

\section{Diagrama Entitate-Relație}
\label{subsec:er_diagram}

Diagrama entitate-relație a bazei de date OLTP este prezentată mai jos:
\begin{figure}[H]
    \centering
    \includegraphics[width=1.0\textwidth]{../schema/TickLy ER Diagram_Initiala.pdf}
    \caption{Diagrama Entitate-Relație a Bazei de Date OLTP}
    \label{fig:er_diagram}
\end{figure}

Această diagramă prezintă:
\begin{itemize}
    \item Toate entitățile
    \item Relațiile între entități (1:1, 1:M, M:N)
\end{itemize}

\section{Diagrama Conceptuală Extinsă}
\label{subsec:conceptual_diagram}

Diagrama conceptuală extinsă a bazei de date OLTP este prezentată Mmai jos:
\begin{figure}[H]
    \centering
    \includegraphics[width=1.0\textwidth]{../schema/TickLy Conceptuală Extinsă.pdf}
    \caption{Diagrama Conceptuală Extinsă a Bazei de Date OLTP}
    \label{fig:conceptual_diagram}
\end{figure}

Baza de date OLTP conține următoarele: 

\subsubsection{Entități independente}

\begin{enumerate}
    \item \textbf{CLIENT} — entitate părinte pentru clienți
    \item \textbf{AGENT} — agenții de suport
    \item \textbf{PRIORITATE} — nivelurile de prioritate ale tichetelor
    \item \textbf{STATUS} — statusurile tichetelor
    \item \textbf{TOPIC} — entitate părinte pentru topic-urile asociate tichetelor
    \item \textbf{CATEGORIE} — categoriile pentru organizarea tichetelor
    \item \textbf{TAG} — tag-uri pentru etichetarea tichetelor
    \item \textbf{DEPARTAMENT} — departamentele organizației
\end{enumerate}

\textbf{Nota:} Categorie este o entitate de referinta catre categoria parinte.

\subsubsection{Entități dependente}

\begin{enumerate}
    \item \textbf{TICKET} — tichetele de suport
    \item \textbf{ATASAMENT} — atașamentele la tichete
    \item \textbf{KB\_ARTICLE} — documentație
    \item \textbf{FEEDBACK} — feedback-urile la tichete
    \item \textbf{SOLUTIE} — soluțiile la tichete
    \item \textbf{ADRESA} — adresele clientilor
\end{enumerate}

\subsubsection{Entități de tip IS-A}

\begin{enumerate}
    \item \textbf{CLIENT $\rightarrow$ CLIENT\_FIZICA / CLIENT\_JURIDICA} — clienți persoane fizice sau juridice
    \item \textbf{TOPIC $\rightarrow$ TOPIC\_SERVICIU / TOPIC\_PRODUS} — topic-uri de tip serviciu sau produs
\end{enumerate}

\subsubsection{Relații Many-to-Many}

Sistemul conține următoarele relații many-to-many (M:N), implementate prin tabele asociative:

\begin{itemize}
    \item \textbf{Ticket $\leftrightarrow$ Agent} (prin \texttt{TICKET\_AGENT}) - un ticket poate fi asignat mai multor agenți, iar un agent poate lucra la mai multe tichete
    \item \textbf{Ticket $\leftrightarrow$ Topic} (prin \texttt{TICKET\_TOPIC}) - un ticket poate fi asociat cu mai multe topic-uri, iar un topic poate fi asociat cu mai multe tichete
    \item \textbf{Agent $\leftrightarrow$ Departament} (prin \texttt{AGENT\_DEPARTAMENT}) - un agent poate aparține mai multor departamente, iar un departament poate avea mai mulți agenți
    \item \textbf{Ticket $\leftrightarrow$ Tag} (prin \texttt{TICKET\_TAG}) - un ticket poate avea mai multe tag-uri, iar un tag poate fi asociat cu mai multe tichete
    \item \textbf{Ticket $\leftrightarrow$ Client} (prin \texttt{COMMENT\_CLIENT}) - un ticket poate avea mai multe comentarii de la client, iar un client poate avea mai multe comentarii la tichete
    \item \textbf{Ticket $\leftrightarrow$ Agent} (prin \texttt{COMMENT\_AGENT}) - un ticket poate avea mai multe comentarii de la agent, iar un agent poate avea mai multe comentarii la tichete
\end{itemize}

\chapter{Diagrama Stea/Fulg a Bazei de Date Depozit}
\label{sec:diagrama_stea}

\begin{figure}[H]
    \centering
    \includegraphics[width=1.0\textwidth]{../schema/TickLy Schema Stea Simplificata.pdf}
    \caption{Diagrama Stea a Bazei de Date Depozit}
    \label{fig:stea_diagram}
\end{figure}

\section{Structura Modelului Stea}

Modelul stea implementat pentru TickLy conține:

\subsubsection{Tabel de Fapte}

\begin{itemize}
    \item \textbf{FACT\_TICKET} - tabelul central de fapte care stochează măsurile și metricile asociate tichetelor
\end{itemize}

\subsubsection{Tabele Dimensiune}

\begin{enumerate}
    \item \textbf{DIM\_CLIENT} - dimensiunea clienților (Type 2 SCD pentru istoricizare)
    \item \textbf{DIM\_AGENT} - dimensiunea agenților (Type 2 SCD)
    \item \textbf{DIM\_DEPARTAMENT} - dimensiunea departamentelor (Type 2 SCD)
    \item \textbf{DIM\_CATEGORIE} - dimensiunea categoriilor (Type 1 SCD)
    \item \textbf{DIM\_TOPIC} - dimensiunea topic-urilor (Type 1 SCD)
    \item \textbf{DIM\_TAG} - dimensiunea tag-urilor (Type 1 SCD)
    \item \textbf{DIM\_TIME} - dimensiunea temporală pentru analize pe perioade
\end{enumerate}

\begin{itemize}
    \item \textbf{Nota - Type 1 SCD:} nu permite urmărirea istoricului modificărilor.
    \item \textbf{Nota - Type 2 SCD:} permite urmărirea istoricului modificărilor.
\end{itemize}

\section{Caracteristici ale Modelului}

\begin{itemize}
    \item \textbf{Slowly Changing Dimensions (SCD)}: Dimensiunile \texttt{DIM\_CLIENT}, \texttt{DIM\_AGENT} și \texttt{DIM\_DEPARTAMENT} sunt implementate ca Type 2 SCD, permițând urmărirea istoricului modificărilor
    \item \textbf{Degenerări}: Status și Prioritate sunt degenerate în tabelul de fapte (stocate direct în \texttt{FACT\_TICKET})
    \item \textbf{Dimensiune Conformată}: Dimensiunea temporală \texttt{DIM\_TIME}
\end{itemize}

\chapter{Descrierea Câmpurilor și Modul de Populare}
\label{sec:campuri_populare}

\section{Tabelul de Fapte: FACT\_TICKET}

\subsubsection{Câmpuri și Sursa de Date}

\begin{longtable}{|>{\raggedright\arraybackslash}p{3.5cm}|>{\raggedright\arraybackslash}p{2.8cm}|>{\raggedright\arraybackslash}p{8.7cm}|}
\hline
\textbf{Câmp} & \textbf{Tip} & \textbf{Mod de Populare din OLTP} \\
\hline
\endfirsthead
\hline
\textbf{Câmp} & \textbf{Tip} & \textbf{Mod de Populare din OLTP} \\
\hline
\endhead
\hline
\endfoot
\hline
\endlastfoot

\texttt{fact\_ticket\_id} & NUMBER & Generat automat (IDENTITY) \\
\hline
\texttt{ticket\_id} & NUMBER & Copiat direct din \texttt{TICKET.ticket\_id} \\
\hline
\texttt{client\_key} & NUMBER & Lookup în \texttt{DIM\_CLIENT} pe baza \texttt{TICKET.client\_id} \\
\hline
\texttt{agent\_key} & NUMBER & Lookup în \texttt{DIM\_AGENT} pe baza \texttt{TICKET\_AGENT.agent\_id} (agentul PRIMARY) \\
\hline
\texttt{departament\_key} & NUMBER & Lookup în \texttt{DIM\_DEPARTAMENT} pe baza \texttt{TICKET.departament\_id} \\
\hline
\texttt{categorie\_key} & NUMBER & Lookup în \texttt{DIM\_CATEGORIE} pe baza \texttt{TICKET.categorie\_id} (poate fi NULL) \\
\hline
\ttwrap{date\_creare\_key} & NUMBER & Lookup în \texttt{DIM\_TIME} pe baza \texttt{TICKET.data\_creare} \\
\hline
\ttwrap{date\_rezolvare\_key} & NUMBER & Lookup în \texttt{DIM\_TIME} pe baza \texttt{TICKET.data\_rezolvare} (poate fi NULL) \\
\hline
\ttwrap{date\_inchidere\_key} & NUMBER & Lookup în \texttt{DIM\_TIME} pe baza \texttt{TICKET.data\_inchidere} (poate fi NULL) \\
\hline
\texttt{status\_id} & NUMBER & Copiat direct din \texttt{TICKET.status\_id} \\
\hline
\texttt{status\_nume} & VARCHAR2 & JOIN cu \texttt{STATUS.nume} \\
\hline
\ttwrap{status\_este\_final} & CHAR(1) & JOIN cu \texttt{STATUS.este\_final} \\
\hline
\texttt{status\_ordine} & NUMBER & Calculat pe baza logicii de business (ordinea statusurilor) \\
\hline
\texttt{prioritate\_id} & NUMBER & Copiat direct din \texttt{TICKET.prioritate\_id} \\
\hline
\texttt{prioritate\_nivel} & NUMBER & JOIN cu \texttt{PRIORITATE.nivel} \\
\hline
\texttt{prioritate\_nume} & VARCHAR2 & JOIN cu \texttt{PRIORITATE.nume} \\
\hline
\ttwrap{prioritate\_timp\_raspuns\_ore} & NUMBER & JOIN cu \texttt{PRIORITATE.timp\_raspuns\_ore} \\
\hline
\texttt{numar\_ticketuri} & NUMBER & Constantă = 1 (granularitatea este un ticket) \\
\hline
\ttwrap{timp\_rezolvare\_ore} & NUMBER & Copiat din \texttt{TICKET.timp\_rezolvare\_ore} sau calculat: \texttt{DATA\_REZOLVARE - DATA\_CREARE} \\
\hline
\ttwrap{timp\_raspuns\_ore} & NUMBER & Calculat pe baza primului comentariu al agentului \\
\hline
\ttwrap{timp\_rezolvare\_minute} & NUMBER & \texttt{timp\_rezolvare\_ore * 60} \\
\hline
\texttt{rating\_feedback} & NUMBER & JOIN cu \texttt{FEEDBACK.rating} (1:1 cu TICKET) \\
\hline
\ttwrap{numar\_comentarii} & NUMBER & COUNT din \texttt{COMMENT\_CLIENT} și \texttt{COMMENT\_AGENT} pentru \texttt{ticket\_id} \\
\hline
\ttwrap{numar\_comentarii\_client} & NUMBER & COUNT din \texttt{COMMENT\_CLIENT} pentru \texttt{ticket\_id} \\
\hline
\ttwrap{numar\_comentarii\_agent} & NUMBER & COUNT din \texttt{COMMENT\_AGENT} pentru \texttt{ticket\_id} \\
\hline
\texttt{numar\_atasamente} & NUMBER & COUNT din \texttt{ATASAMENT} pentru \texttt{ticket\_id} \\
\hline
\texttt{cost\_estimativ} & NUMBER(10,2) & Calculat pe baza \texttt{TOPIC.tarif} sau \texttt{TOPIC.pret} asociat \\
\hline
\texttt{load\_date} & DATE & SYSDATE la momentul încărcării \\
\hline
\end{longtable}

\section{Dimensiunea: DIM\_CLIENT}

\begin{longtable}{|>{\raggedright\arraybackslash}p{3.5cm}|>{\raggedright\arraybackslash}p{2.8cm}|>{\raggedright\arraybackslash}p{8.7cm}|}
\hline
\textbf{Câmp} & \textbf{Tip} & \textbf{Mod de Populare din OLTP} \\
\hline
\endfirsthead
\hline
\textbf{Câmp} & \textbf{Tip} & \textbf{Mod de Populare din OLTP} \\
\hline
\endhead
\hline
\endfoot
\hline
\endlastfoot

\texttt{client\_key} & NUMBER & Generat automat (IDENTITY) \\
\hline
\texttt{client\_id} & NUMBER & Copiat din \texttt{CLIENT.client\_id} \\
\hline
\texttt{email} & VARCHAR2 & Copiat din \texttt{CLIENT.email} \\
\hline
\texttt{phone} & VARCHAR2 & Copiat din \texttt{CLIENT.phone} \\
\hline
\ttwrap{registration\_date} & DATE & Copiat din \texttt{CLIENT.registration\_date} \\
\hline
\texttt{client\_type} & CHAR(1) & Copiat din \texttt{CLIENT.client\_type} ('F' sau 'J') \\
\hline
\texttt{nume} & VARCHAR2 & Copiat din \texttt{CLIENT\_FIZICA.nume} (dacă \texttt{client\_type = 'F'}) \\
\hline
\texttt{prenume} & VARCHAR2 & Copiat din \texttt{CLIENT\_FIZICA.prenume} (dacă \texttt{client\_type = 'F'}) \\
\hline
\texttt{cnp} & VARCHAR2 & Copiat din \texttt{CLIENT\_FIZICA.cnp} (dacă \texttt{client\_type = 'F'}) \\
\hline
\texttt{denumire} & VARCHAR2 & Copiat din \texttt{CLIENT\_JURIDICA.denumire} (dacă \texttt{client\_type = 'J'}) \\
\hline
\texttt{cui} & VARCHAR2 & Copiat din \texttt{CLIENT\_JURIDICA.cui} (dacă \texttt{client\_type = 'J'}) \\
\hline
\texttt{sediu\_social} & VARCHAR2 & Copiat din \texttt{CLIENT\_JURIDICA.sediu\_social} (dacă \texttt{client\_type = 'J'}) \\
\hline
\ttwrap{reprezentant\_legal} & VARCHAR2 & Copiat din \texttt{CLIENT\_JURIDICA.reprezentant\_legal} (dacă \texttt{client\_type = 'J'}) \\
\hline
\texttt{is\_active} & CHAR(1) & Calculat pe baza existenței tichetelor recente sau flag explicit \\
\hline
\texttt{valid\_from} & DATE & SYSDATE pentru înregistrări noi, sau data modificării pentru SCD Type 2 \\
\hline
\texttt{valid\_to} & DATE & NULL pentru înregistrări curente, sau data modificării pentru înregistrări istorice \\
\hline
\texttt{is\_current} & CHAR(1) & 'Y' dacă \texttt{valid\_to IS NULL}, altfel 'N' \\
\hline
\texttt{load\_date} & DATE & SYSDATE la momentul încărcării \\
\hline
\end{longtable}

\section{Dimensiunea: DIM\_AGENT}

\begin{longtable}{|>{\raggedright\arraybackslash}p{3.5cm}|>{\raggedright\arraybackslash}p{2.8cm}|>{\raggedright\arraybackslash}p{8.7cm}|}
\hline
\textbf{Câmp} & \textbf{Tip} & \textbf{Mod de Populare din OLTP} \\
\hline
\endfirsthead
\hline
\textbf{Câmp} & \textbf{Tip} & \textbf{Mod de Populare din OLTP} \\
\hline
\endhead
\hline
\endfoot
\hline
\endlastfoot

\texttt{agent\_key} & NUMBER & Generat automat (IDENTITY) \\
\hline
\texttt{agent\_id} & NUMBER & Copiat din \texttt{AGENT.agent\_id} \\
\hline
\texttt{nume} & VARCHAR2 & Copiat din \texttt{AGENT.nume} \\
\hline
\texttt{prenume} & VARCHAR2 & Copiat din \texttt{AGENT.prenume} \\
\hline
\ttwrap{nume\_complet} & VARCHAR2 & Concatenare: \texttt{AGENT.nume || ' ' || AGENT.prenume} \\
\hline
\texttt{email} & VARCHAR2 & Copiat din \texttt{AGENT.email} \\
\hline
\texttt{telefon} & VARCHAR2 & Copiat din \texttt{AGENT.telefon} \\
\hline
\texttt{hire\_date} & DATE & Copiat din \texttt{AGENT.hire\_date} \\
\hline
\texttt{is\_active} & CHAR(1) & Copiat din \texttt{AGENT.is\_active} \\
\hline
\ttwrap{ani\_experienta} & NUMBER & Calculat: \texttt{TRUNC(MONTHS\_BETWEEN(SYSDATE, AGENT.hire\_date) / 12)} \\
\hline
\texttt{valid\_from} & DATE & SYSDATE pentru înregistrări noi, sau data modificării pentru SCD Type 2 \\
\hline
\texttt{valid\_to} & DATE & NULL pentru înregistrări curente, sau data modificării pentru înregistrări istorice \\
\hline
\texttt{is\_current} & CHAR(1) & 'Y' dacă \texttt{valid\_to IS NULL}, altfel 'N' \\
\hline
\texttt{load\_date} & DATE & SYSDATE la momentul încărcării \\
\hline
\end{longtable}

\section{Dimensiunea: DIM\_DEPARTAMENT}

\begin{longtable}{|>{\raggedright\arraybackslash}p{3.5cm}|>{\raggedright\arraybackslash}p{2.8cm}|>{\raggedright\arraybackslash}p{8.7cm}|}
\hline
\textbf{Câmp} & \textbf{Tip} & \textbf{Mod de Populare din OLTP} \\
\hline
\endfirsthead
\hline
\textbf{Câmp} & \textbf{Tip} & \textbf{Mod de Populare din OLTP} \\
\hline
\endhead
\hline
\endfoot
\hline
\endlastfoot

\texttt{departament\_key} & NUMBER & Generat automat (IDENTITY) \\
\hline
\texttt{departament\_id} & NUMBER & Copiat din \texttt{DEPARTAMENT.departament\_id} \\
\hline
\texttt{nume} & VARCHAR2 & Copiat din \texttt{DEPARTAMENT.nume} \\
\hline
\texttt{descriere} & VARCHAR2 & Copiat din \texttt{DEPARTAMENT.descriere} \\
\hline
\ttwrap{manager\_nume} & VARCHAR2 & JOIN cu \texttt{AGENT} pe \texttt{DEPARTAMENT.manager\_id}: \texttt{AGENT.nume || ' ' || AGENT.prenume} \\
\hline
\texttt{manager\_email} & VARCHAR2 & JOIN cu \texttt{AGENT} pe \texttt{DEPARTAMENT.manager\_id}: \texttt{AGENT.email} \\
\hline
\texttt{numar\_agenti} & NUMBER & COUNT din \texttt{AGENT\_DEPARTAMENT} pentru \texttt{departament\_id} unde \texttt{este\_principal = 'Y'} \\
\hline
\texttt{valid\_from} & DATE & SYSDATE pentru înregistrări noi, sau data modificării pentru SCD Type 2 \\
\hline
\texttt{valid\_to} & DATE & NULL pentru înregistrări curente, sau data modificării pentru înregistrări istorice \\
\hline
\texttt{is\_current} & CHAR(1) & 'Y' dacă \texttt{valid\_to IS NULL}, altfel 'N' \\
\hline
\texttt{load\_date} & DATE & SYSDATE la momentul încărcării \\
\hline
\end{longtable}

\section{Dimensiunea: DIM\_CATEGORIE}

\begin{longtable}{|>{\raggedright\arraybackslash}p{3.5cm}|>{\raggedright\arraybackslash}p{2.8cm}|>{\raggedright\arraybackslash}p{8.7cm}|}
\hline
\textbf{Câmp} & \textbf{Tip} & \textbf{Mod de Populare din OLTP} \\
\hline
\endfirsthead
\hline
\textbf{Câmp} & \textbf{Tip} & \textbf{Mod de Populare din OLTP} \\
\hline
\endhead
\hline
\endfoot
\hline
\endlastfoot

\texttt{categorie\_key} & NUMBER & Generat automat (IDENTITY) \\
\hline
\texttt{categorie\_id} & NUMBER & Copiat din \texttt{CATEGORIE.categorie\_id} \\
\hline
\texttt{nume} & VARCHAR2 & Copiat din \texttt{CATEGORIE.nume} \\
\hline
\texttt{descriere} & VARCHAR2 & Copiat din \texttt{CATEGORIE.descriere} \\
\hline
\ttwrap{categorie\_parinte\_id} & NUMBER & Copiat din \texttt{CATEGORIE.categorie\_parinte\_id} \\
\hline
\ttwrap{categorie\_parinte\_nume} & VARCHAR2 & Self-join cu \texttt{CATEGORIE} pe \texttt{categorie\_parinte\_id} pentru a obține numele părinte \\
\hline
\texttt{nivel\_ierarhie} & NUMBER & Calculat recursiv: nivelul în ierarhia categoriilor (1 = rădăcină) \\
\hline
\ttwrap{categorie\_completa} & VARCHAR2 & Concatenare a întregii ierarhii: "Categorie Părinte > Categorie Curentă" \\
\hline
\texttt{load\_date} & DATE & SYSDATE la momentul încărcării \\
\hline
\end{longtable}

\section{Dimensiunea: DIM\_TOPIC}

\begin{longtable}{|>{\raggedright\arraybackslash}p{3.5cm}|>{\raggedright\arraybackslash}p{2.8cm}|>{\raggedright\arraybackslash}p{8.7cm}|}
\hline
\textbf{Câmp} & \textbf{Tip} & \textbf{Mod de Populare din OLTP} \\
\hline
\endfirsthead
\hline
\textbf{Câmp} & \textbf{Tip} & \textbf{Mod de Populare din OLTP} \\
\hline
\endhead
\hline
\endfoot
\hline
\endlastfoot

\texttt{topic\_key} & NUMBER & Generat automat (IDENTITY) \\
\hline
\texttt{topic\_id} & NUMBER & Copiat din \texttt{TOPIC.topic\_id} \\
\hline
\texttt{nume} & VARCHAR2 & Copiat din \texttt{TOPIC.nume} \\
\hline
\texttt{descriere} & VARCHAR2 & Copiat din \texttt{TOPIC.descriere} \\
\hline
\texttt{topic\_type} & CHAR(1) & Copiat din \texttt{TOPIC.topic\_type} ('S' sau 'P') \\
\hline
\texttt{tip\_serviciu} & VARCHAR2 & Copiat din \texttt{TOPIC\_SERVICIU.tip\_serviciu} (dacă \texttt{topic\_type = 'S'}) \\
\hline
\ttwrap{durata\_estimata} & NUMBER & Copiat din \texttt{TOPIC\_SERVICIU.durata\_estimata} (dacă \texttt{topic\_type = 'S'}) \\
\hline
\texttt{tarif} & NUMBER(10,2) & Copiat din \texttt{TOPIC\_SERVICIU.tarif} (dacă \texttt{topic\_type = 'S'}) \\
\hline
\texttt{versiune} & VARCHAR2 & Copiat din \texttt{TOPIC\_PRODUS.versiune} (dacă \texttt{topic\_type = 'P'}) \\
\hline
\texttt{pret} & NUMBER(10,2) & Copiat din \texttt{TOPIC\_PRODUS.pret} (dacă \texttt{topic\_type = 'P'}) \\
\hline
\texttt{stoc} & NUMBER & Copiat din \texttt{TOPIC\_PRODUS.stoc} (dacă \texttt{topic\_type = 'P'}) \\
\hline
\texttt{load\_date} & DATE & SYSDATE la momentul încărcării \\
\hline
\end{longtable}

\section{Dimensiunea: DIM\_TAG}

\begin{longtable}{|>{\raggedright\arraybackslash}p{3.5cm}|>{\raggedright\arraybackslash}p{2.8cm}|>{\raggedright\arraybackslash}p{8.7cm}|}
\hline
\textbf{Câmp} & \textbf{Tip} & \textbf{Mod de Populare din OLTP} \\
\hline
\endfirsthead
\hline
\textbf{Câmp} & \textbf{Tip} & \textbf{Mod de Populare din OLTP} \\
\hline
\endhead
\hline
\endfoot
\hline
\endlastfoot

\texttt{tag\_key} & NUMBER & Generat automat (IDENTITY) \\
\hline
\texttt{tag\_id} & NUMBER & Copiat din \texttt{TAG.tag\_id} \\
\hline
\texttt{nume} & VARCHAR2 & Copiat din \texttt{TAG.nume} \\
\hline
\texttt{culoare} & VARCHAR2 & Copiat din \texttt{TAG.culoare} \\
\hline
\texttt{descriere} & VARCHAR2 & Copiat din \texttt{TAG.descriere} \\
\hline
\texttt{load\_date} & DATE & SYSDATE la momentul încărcării \\
\hline
\end{longtable}

\section{Dimensiunea: DIM\_TIME}

Dimensiunea temporală este populată prin proces ETL separat care generează toate datele dintr-un interval specificat (de exemplu, 2000-2050). Fiecare câmp este calculat pe baza datei complete:

\begin{longtable}{|>{\raggedright\arraybackslash}p{3.5cm}|>{\raggedright\arraybackslash}p{2.8cm}|>{\raggedright\arraybackslash}p{8.7cm}|}
\hline
\textbf{Câmp} & \textbf{Tip} & \textbf{Mod de Populare} \\
\hline
\endfirsthead
\hline
\textbf{Câmp} & \textbf{Tip} & \textbf{Mod de Populare} \\
\hline
\endhead
\hline
\endfoot
\hline
\endlastfoot

\texttt{date\_key} & NUMBER & Format: YYYYMMDD (ex: 20250127) \\
\hline
\texttt{data\_completa} & DATE & Data completă \\
\hline
\texttt{an} & NUMBER(4) & \texttt{EXTRACT(YEAR FROM data\_completa)} \\
\hline
\texttt{trimestru} & NUMBER(1) & \texttt{TO\_NUMBER(TO\_CHAR(data\_completa, 'Q'))} \\
\hline
\texttt{luna} & NUMBER(2) & \texttt{EXTRACT(MONTH FROM data\_completa)} \\
\hline
\texttt{luna\_nume} & VARCHAR2 & \texttt{TO\_CHAR(data\_completa, 'Month')} \\
\hline
\texttt{luna\_abrev} & VARCHAR2 & \texttt{TO\_CHAR(data\_completa, 'Mon')} \\
\hline
\texttt{zi} & NUMBER(2) & \texttt{EXTRACT(DAY FROM data\_completa)} \\
\hline
\texttt{saptamana\_an} & NUMBER(2) & \texttt{TO\_NUMBER(TO\_CHAR(data\_completa, 'WW'))} \\
\hline
\ttwrap{zi\_saptamana} & NUMBER(1) & \texttt{TO\_NUMBER(TO\_CHAR(data\_completa, 'D'))} (1=Luni, 7=Duminică) \\
\hline
\ttwrap{zi\_saptamana\_nume} & VARCHAR2 & \texttt{TO\_CHAR(data\_completa, 'Day')} \\
\hline
\ttwrap{este\_weekend} & CHAR(1) & 'Y' dacă \texttt{zi\_saptamana IN (6,7)}, altfel 'N' \\
\hline
\ttwrap{este\_sarbatoare} & CHAR(1) & Verificare în tabel de sărbători legale \\
\hline
\ttwrap{nume\_sarbatoare} & VARCHAR2 & Numele sărbătorii (dacă există) \\
\hline
\ttwrap{zi\_lucratoare} & CHAR(1) & 'Y' dacă este zi lucrătoare, 'N' dacă este weekend sau sărbătoare (consistență cu \texttt{este\_weekend} și \texttt{este\_sarbatoare}) \\
\hline
\end{longtable}

\chapter{Constrângerile Specifice Depozitelor de Date}
\label{sec:constrainturi}

\section{Constrângeri de Integritate Referențială}

Toate tabelele dimensiune și tabelul de fapte folosesc constrângeri de cheie străină (FK) pentru a menține integritatea referențială:

\begin{itemize}
    \item \textbf{FK\_FACT\_CLIENT}: Asigură că fiecare ticket din \texttt{FACT\_TICKET} are un client valid în \texttt{DIM\_CLIENT}
    \item \textbf{FK\_FACT\_AGENT}: Asigură că fiecare ticket are un agent valid în \texttt{DIM\_AGENT}
    \item \textbf{FK\_FACT\_DEPARTAMENT}: Asigură că fiecare ticket aparține unui departament valid
    \item \textbf{FK\_FACT\_CATEGORIE}: Asigură că categoria (dacă există) este validă
    \item \textbf{FK\_FACT\_DATE\_CREARE}: Asigură că data de creare este validă în \texttt{DIM\_TIME}
    \item \textbf{FK\_FACT\_DATE\_REZOLVARE}: Asigură că data de rezolvare este validă în \texttt{DIM\_TIME}
    \item \textbf{FK\_FACT\_DATE\_INCHIDERE}: Asigură că data de închidere este validă în \texttt{DIM\_TIME}
\end{itemize}

\section{Constrângeri UNIQUE}

\begin{itemize}
    \item \textbf{UK\_FACT\_TICKET\_ID}: Asigură că fiecare \texttt{ticket\_id} apare o singură dată în \texttt{FACT\_TICKET} (prevenirea duplicatelor)
    \item \textbf{UK\_DIM\_CLIENT\_ID}: Asigură unicitatea combinației \texttt{(client\_id, valid\_from)} pentru SCD Type 2
    \item \textbf{UK\_DIM\_AGENT\_ID}: Asigură unicitatea combinației \texttt{(agent\_id, valid\_from)} pentru SCD Type 2
    \item \textbf{UK\_DIM\_DEPARTAMENT\_ID}: Asigură unicitatea combinației \texttt{(departament\_id, valid\_from)} pentru SCD Type 2
    \item \textbf{UK\_DIM\_TIME\_DATA}: Asigură că fiecare dată apare o singură dată în \texttt{DIM\_TIME}
\end{itemize}

\section{Constrângeri CHECK}

\begin{itemize}
    \item \textbf{CHECK pentru DIM\_TIME}:
    \begin{itemize}
        \item \texttt{trimestru BETWEEN 1 AND 4}
        \item \texttt{luna BETWEEN 1 AND 12}
        \item \texttt{zi BETWEEN 1 AND 31}
        \item \texttt{saptamana\_an BETWEEN 1 AND 53}
        \item \texttt{zi\_saptamana BETWEEN 1 AND 7}
        \item \texttt{este\_weekend IN ('Y', 'N')}
        \item \texttt{este\_sarbatoare IN ('Y', 'N')}
        \item \texttt{zi\_lucratoare IN ('Y', 'N')}
    \end{itemize}
    \item \textbf{CHECK pentru FACT\_TICKET}:
    \begin{itemize}
        \item \texttt{rating\_feedback BETWEEN 1 AND 5}
    \end{itemize}
\end{itemize}

\section{Justificarea Constrângerilor}

\begin{enumerate}
    \item \textbf{Constrângerile FK} previn inserarea de înregistrări orfane în tabelul de fapte și asigură că toate referințele la dimensiuni sunt valide.
    
    \item \textbf{Constrângerile UNIQUE} Mai multe înregistrări ale aceleiași entități cu intervale de valabilitate diferite sunt permise pentru SCD Type 2, ceea ce permite o istoricizare precisă a modificărilor.
    
    \item \textbf{Constrângerile CHECK} asigură validitatea datelor la nivel de aplicație, prevenind erorile de logică (spre exemplu: trimestru 5, luna 13, rating 6).
    
    \item \textbf{UK\_FACT\_TICKET\_ID} este critică pentru a preveni duplicarea tichetelor în procesul ETL, asigurând că fiecare ticket din OLTP apare exact o dată în DW.
\end{enumerate}

\chapter{Indecșii Specifici Depozitelor de Date}
\label{sec:indecsi}

\section{Indecși B-Tree}

\subsubsection{Pe Tabelul de Fapte (FACT\_TICKET)}

\begin{itemize}
    \item \textbf{IDX\_FACT\_CLIENT}: Pe \texttt{client\_key} - pentru join-uri rapide cu dimensiunea clienților
    \item \textbf{IDX\_FACT\_AGENT}: Pe \texttt{agent\_key} - pentru analize pe agenți
    \item \textbf{IDX\_FACT\_DEPARTAMENT}: Pe \texttt{departament\_key} - pentru analize pe departamente
    \item \textbf{IDX\_FACT\_CATEGORIE}: Pe \texttt{categorie\_key} - pentru filtrare pe categorii
    \item \textbf{IDX\_FACT\_DATE\_CREARE}: Pe \texttt{date\_creare\_key} - pentru analize temporale (cel mai frecvent folosit)
    \item \textbf{IDX\_FACT\_DATE\_REZOLVARE}: Pe \texttt{date\_rezolvare\_key} - pentru analize pe perioada de rezolvare
    \item \textbf{IDX\_FACT\_STATUS}: Pe \texttt{status\_id} - pentru filtrare pe status
    \item \textbf{IDX\_FACT\_PRIORITATE}: Pe \texttt{prioritate\_id} - pentru filtrare pe prioritate
\end{itemize}

\subsubsection{Pe Tabelele Dimensiune}

\begin{itemize}
    \item \textbf{IDX\_DIM\_CLIENT\_ID}: Pe \texttt{client\_id} - pentru lookup rapid în procesul ETL
    \item \textbf{IDX\_DIM\_CLIENT\_CURRENT}: Pe \texttt{is\_current} - pentru a găsi rapid versiunea curentă (SCD Type 2)
    \item \textbf{IDX\_DIM\_AGENT\_ID}: Pe \texttt{agent\_id} - pentru lookup în ETL
    \item \textbf{IDX\_DIM\_AGENT\_CURRENT}: Pe \texttt{is\_current} - pentru versiunea curentă
    \item \textbf{IDX\_DIM\_DEPARTAMENT\_ID}: Pe \texttt{departament\_id} - pentru lookup în ETL
    \item \textbf{IDX\_DIM\_DEPARTAMENT\_CURRENT}: Pe \texttt{is\_current} - pentru versiunea curentă
    \item \textbf{IDX\_DIM\_TIME\_AN}: Pe \texttt{an} - pentru analize anuale
    \item \textbf{IDX\_DIM\_TIME\_LUNA}: Pe \texttt{(an, luna)} - pentru analize lunare
    \item \textbf{IDX\_DIM\_TIME\_TRIMESTRU}: Pe \texttt{(an, trimestru)} - pentru analize trimestriale
\end{itemize}

\section{Indecși Bitmap}

% Indecșii bitmap sunt optimizați pentru coloane cu cardinalitate redusă (puține valori distincte).

\subsubsection{Pe Tabelul de Fapte}

\begin{itemize}
    \item \textbf{BMP\_FACT\_STATUS\_NUME}: Pe \texttt{status\_nume} - pentru filtrare rapidă pe status (ex: "Deschis", "În Progres", "Rezolvat")
    \item \textbf{BMP\_FACT\_STATUS\_FINAL}: Pe \texttt{status\_este\_final} - pentru a identifica rapid tichetele finalizate ('Y'/'N')
    \item \textbf{BMP\_FACT\_PRIORITATE\_NIVEL}: Pe \texttt{prioritate\_nivel} - pentru filtrare pe nivel de prioritate (1-5)
    \item \textbf{BMP\_FACT\_PRIORITATE\_NUME}: Pe \texttt{prioritate\_nume} - pentru filtrare pe nume prioritate
    \item \textbf{BMP\_FACT\_RATING}: Pe \texttt{rating\_feedback} - pentru analize pe rating (1-5)
\end{itemize}

\subsubsection{Pe Tabelele Dimensiune}

\begin{itemize}
    \item \textbf{BMP\_DIM\_CLIENT\_TYPE}: Pe \texttt{client\_type} - pentru filtrare pe tip client ('F'/'J')
    \item \textbf{BMP\_DIM\_CLIENT\_ACTIVE}: Pe \texttt{is\_active} - pentru a identifica clienții activi
    \item \textbf{BMP\_DIM\_CLIENT\_CURRENT}: Pe \texttt{is\_current} - pentru versiunea curentă (SCD Type 2)
    \item \textbf{BMP\_DIM\_AGENT\_ACTIVE}: Pe \texttt{is\_active} - pentru agenții activi
    \item \textbf{BMP\_DIM\_AGENT\_CURRENT}: Pe \texttt{is\_current} - pentru versiunea curentă
    \item \textbf{BMP\_DIM\_DEPARTAMENT\_CURRENT}: Pe \texttt{is\_current} - pentru versiunea curentă
    \item \textbf{BMP\_DIM\_TOPIC\_TYPE}: Pe \texttt{topic\_type} - pentru filtrare pe tip topic ('S'/'P')
    \item \textbf{BMP\_DIM\_TIME\_WEEKEND}: Pe \texttt{este\_weekend} - pentru analize pe weekend
    \item \textbf{BMP\_DIM\_TIME\_SARBATOARE}: Pe \texttt{este\_sarbatoare} - pentru analize pe sărbători
    \item \textbf{BMP\_DIM\_TIME\_TRIMESTRU}: Pe \texttt{trimestru} - pentru analize trimestriale
    \item \textbf{BMP\_DIM\_TIME\_LUNA}: Pe \texttt{luna} - pentru analize lunare
    \item \textbf{BMP\_DIM\_TIME\_ZI\_SAPTAMANA}: Pe \texttt{zi\_saptamana} - pentru analize pe zile ale săptămânii
\end{itemize}

\section{Cerere în Limbaj Natural care Utilizează Indecșii Specificați}

\textbf{Cerere}: "Să se afișeze numărul total de tichete rezolvate cu rating 5 stele, create în weekend-urile din trimestrul 1 al anului 2025, grupate pe departamente și agenți activi."

Această cerere va utiliza următorii indecși:
\begin{itemize}
    \item \textbf{BMP\_FACT\_RATING}: Pentru filtrarea rapidă a tichetelor cu \texttt{rating\_feedback = 5}
    \item \textbf{BMP\_FACT\_STATUS\_FINAL}: Pentru filtrarea tichetelor finalizate (\texttt{status\_este\_final = 'Y'})
    \item \textbf{BMP\_DIM\_TIME\_WEEKEND}: Pentru identificarea rapidă a zilelor de weekend (\texttt{este\_weekend = 'Y'})
    \item \textbf{BMP\_DIM\_TIME\_TRIMESTRU}: Pentru filtrarea pe trimestrul 1
    \item \textbf{IDX\_DIM\_TIME\_AN}: Pentru filtrarea pe anul 2025
    \item \textbf{IDX\_FACT\_DATE\_CREARE}: Pentru join-ul eficient cu dimensiunea temporală
    \item \textbf{IDX\_FACT\_DEPARTAMENT}: Pentru join-ul cu dimensiunea departamentelor
    \item \textbf{IDX\_FACT\_AGENT}: Pentru join-ul cu dimensiunea agenților
    \item \textbf{BMP\_DIM\_AGENT\_ACTIVE}: Pentru filtrarea agenților activi
    \item \textbf{BMP\_DIM\_AGENT\_CURRENT}: Pentru a obține versiunea curentă a agenților
\end{itemize}

