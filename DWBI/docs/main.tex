% Sablon pentru realizarea lucrarii de licenta, conform cu recomandarile
% din ghidul de redactare:
% - https://fmi.unibuc.ro/finalizare-studii/
% - https://drive.google.com/file/d/1xj9kZZgTkcKMJkMLRuoYRgLQ1O8CX0mv/view

% Multumiri lui Gabriel Majeri, acest sablon a fost creat pe baza
% codului sursa a lucrarii sale de licenta. 
% Codul sursa: https://github.com/GabrielMajeri/bachelors-thesis
% Website: https://www.gabrielmajeri.ro/
%
% Aceast sablon este licentiat sub Creative Commons Attribution 4.0 International License.

\documentclass[12pt, a4paper]{report}

% Suport pentru diacritice și alte simboluri
\usepackage{fontspec}

% Suport pentru mai multe limbi
\usepackage{polyglossia}

% Setează limba textului la română
\setdefaultlanguage{romanian}
% Am nevoie de engleză pentru rezumat
\setotherlanguages{english}

% Indentează și primul paragraf al fiecărei noi secțiuni
\SetLanguageKeys{romanian}{indentfirst=true}

% Suport pentru diferite stiluri de ghilimele
\usepackage{csquotes}

\DeclareQuoteStyle{romanian}
  {\quotedblbase}
  {\textquotedblright}
  {\guillemotleft}
  {\guillemotright}

% Utilizează biblatex pentru referințe bibliografice
\usepackage[
    maxbibnames=50,
    sorting=nty
]{biblatex}

% \addbibresource{bibliography.bib}  % decomentează dacă există bibliography.bib

% Setează spațiere inter-linie la 1.5
\usepackage{setspace}
\onehalfspacing

% Modificarea geometriei paginii
\usepackage{geometry}

% Include funcțiile de grafică
\usepackage{graphicx}
% Încarcă imaginile din directorul `assets` și din schema (pentru raport)
\graphicspath{{./assets/}{../schema/}}

% Listări de cod
\usepackage{listings}

% Linkuri interactive în PDF
\usepackage[
    colorlinks,
    linkcolor={black},
    menucolor={black},
    citecolor={black},
    urlcolor={blue}
]{hyperref}

% Simboluri matematice codificate Unicode
\usepackage[warnings-off={mathtools-colon,mathtools-overbracket}]{unicode-math}

% Comenzi matematice
\usepackage{amsmath}
\usepackage{mathtools}

\usepackage{listings}
\usepackage{xcolor} % Pentru culori personalizate

\lstset{
    language=SQL,
    basicstyle=\ttfamily\small,
    keywordstyle=\color{blue}\bfseries,
    commentstyle=\color{green!40!black},
    stringstyle=\color{orange},
    numbers=left,
    numberstyle=\tiny\color{gray},
    stepnumber=1,
    numbersep=5pt,
    breaklines=true,
    breakatwhitespace=true,
    frame=single,
    backgroundcolor=\color{gray!5},
    captionpos=b,
    tabsize=2,
    showspaces=false,
    showstringspaces=false
}

% Formule matematice
\newcommand{\bigO}[1]{\symcal{O}\left(#1\right)}
\DeclarePairedDelimiter\abs{\lvert}{\rvert}

% Suport pentru rezumat în două limbi
% Bazat pe https://tex.stackexchange.com/a/70818
\newenvironment{abstractpage}
  {\cleardoublepage\vspace*{\fill}\thispagestyle{empty}}
  {\vfill\cleardoublepage}
\renewenvironment{abstract}[1]
  {\bigskip\selectlanguage{#1}%
   \begin{center}\bfseries\abstractname\end{center}}
  {\par\bigskip}

% Suport pentru anexe
\usepackage{appendix}

% Stiluri diferite de headere și footere
\usepackage{fancyhdr}

% Pachete necesare pentru raportul de analiză (tabele, liste, figuri)
\usepackage{float}
\usepackage{longtable}
\usepackage{booktabs}
\usepackage{enumitem}

% Metadate — proiect TickLy DW\&BI
\title{Data Warehouse și Business Intelligence\\
\large TickLy}
\author{AMAlytics \\ Murariu Andrei, Mitroica Matei, Cornea Alexandru}

% Generează variabilele cu @
\makeatletter

\begin{document}

% Front matter
\cleardoublepage
\let\ps@plain

% Pagina de titlu
\restoregeometry
\newgeometry{margin=2.5cm}

\fancypagestyle{main}{
  \fancyhf{}
  \renewcommand\headrulewidth{0pt}
  \fancyhead[C]{}
  \fancyfoot[C]{\thepage}
}

\addtocounter{page}{1}

% Pagina de titlu (cu \@title și \@author)
\begin{titlepage}

% Redu marginile
\newgeometry{left=2cm,right=2cm,bottom=1cm}

\vspace{1cm}

\begin{center}
\Large \textbf{Raport de analiză}
\end{center}

\vspace{0.5cm}

\begin{center}
\huge \textbf{\@title}
\end{center}

\vspace{2cm}

\begin{center}
\large \textbf{Echipă \\ \@author}
\end{center}

\vspace{0.25cm}

\vspace{1cm}

\end{titlepage}
\restoregeometry
\newgeometry{margin=2.5cm}

% Rezumatul
\begin{abstractpage}

\begin{abstract}{romanian}
Acest raport descrie analiza modelului de date pentru TickLy, un sistem de management al tichetelor de suport, în contextul proiectului de Data Warehouse și Business Intelligence. Prezentăm modelul entitate–relație extins al bazei OLTP, diagrama conceptuală extinsă și modelul stea al depozitului, cu un tabel de fapte (\texttt{FACT\_TICKET}) și cinci tabele dimensiune. Detaliem câmpurile fiecărui tabel din DW și modul de populare din OLTP, constrângerile și indecșii specifici depozitelor de date, obiectele de tip dimensiune, partiționarea tabelelor, o cerere SQL complexă cu tehnici de optimizare și cinci cereri specifice DW, cu grad diferit de complexitate, concretizate în rapoarte și grafice.
\end{abstract}

\end{abstractpage}


\tableofcontents

% Main matter
\cleardoublepage
\pagestyle{main}
\let\ps@plain\ps@main

% \printbibliography[heading=bibintoc]  % decomentează dacă folosești bibliography.bib

\end{document}