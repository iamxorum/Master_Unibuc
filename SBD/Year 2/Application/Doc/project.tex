\chapter{CareConnect} 

\section{Prezentarea modelului proiectat}

Arhiva repository: https://github.com/iamxorum/Master\_Unibuc

Baza de date CareConnect este un sistem medical care gestionează informații despre pacienți, personalul medical și fișele medicale asociate. Modelul este proiectat pentru a demonstra mecanisme avansate de securitate în Oracle Database, respectând principiile de confidențialitate și integritate a datelor medicale sensibile.

Sistemul este structurat pe patru entități principale:

\begin{itemize}
    \item \textbf{DEPARTAMENT} - reprezintă departamentele medicale ale instituției (ex: Cardiologie, Neurologie)
    \item \textbf{PERSONAL\_MEDICAL} - angajații sistemului medical (medici, asistenți, personal de recepție, administratori)
    \item \textbf{PACIENT} - pacienții înregistrați în sistem
    \item \textbf{FISA\_MEDICALA} - fișele medicale asociate pacienților, create de medici
\end{itemize}

\subsection{Reguli de business}

Modelul respectă următoarele reguli de business:

\begin{enumerate}
    \item Fiecare pacient are un CNP unic, care este criptat în baza de date folosind AES-256
    \item Fiecare fișă medicală este asociată unui pacient și unui medic responsabil
    \item Fișele medicale au un nivel de confidențialitate (1, 2 sau 3) care determină accesul utilizatorilor
    \item Personalul medical este organizat pe departamente și are un rol (MEDIC, ASISTENT, RECEPȚIE, ADMIN) cu un grad de acces corespunzător (1-4)
    \item Fiecare membru al personalului medical are un cont Oracle Database asociat pentru autentificare
\end{enumerate}

\section{Diagrama conceptuală}

Diagrama conceptuală a modelului este prezentată în Figura.

\begin{figure}[h]
    \centering
    \includegraphics[width=0.5\textwidth]{images/assets/DB Diagram.png}
    \caption{Diagrama CareConnect}
    \label{fig:diagrama-conceptuala}
\end{figure}

Aceasta ilustrează entitățile principale și relațiile dintre ele:

\begin{itemize}
    \item \textbf{DEPARTAMENT} $\leftrightarrow$ \textbf{PERSONAL\_MEDICAL}: relație 1:N (un departament are mai mulți angajați)
    \item \textbf{PERSONAL\_MEDICAL} $\leftrightarrow$ \textbf{FISA\_MEDICALA}: relație 1:N (un medic poate crea mai multe fișe)
    \item \textbf{PACIENT} $\leftrightarrow$ \textbf{FISA\_MEDICALA}: relație 1:N (un pacient poate avea mai multe fișe medicale)
\end{itemize}

\section{Schemele relaționale}

Schema relațională a bazei de date este următoarea:

\begin{description}
    \item[\textbf{DEPARTAMENT}]
     
    \item[\textbf{PERSONAL\_MEDICAL}]
    \begin{itemize}
        \item FK: id\_departament $\rightarrow$ DEPARTAMENT(id\_departament)
        \item Constraint: rol $\in$ \{'MEDIC', 'ASISTENT', 'ADMIN', 'RECEPTIE'\}
        \item Constraint: grad\_acces $\in$ \{0, 1, 2, 3, 4\}
    \end{itemize}
    
    \item[\textbf{PACIENT}]
    \begin{itemize}
        \item Constraint: sex $\in$ \{'M', 'F'\}
        \item Constraint: grupa\_sanguina $\in$ \{'A+', 'A-', 'B+', 'B-', 'AB+', 'AB-', 'O+', 'O-'\}
        \item Notă: cnp este de tip RAW(100) - stocat criptat
    \end{itemize}
    
    \item[\textbf{FISA\_MEDICALA}]
    \begin{itemize}
        \item FK: id\_pacient $\rightarrow$ PACIENT(id\_pacient)
        \item FK: id\_medic $\rightarrow$ PERSONAL\_MEDICAL(id\_personal)
        \item Constraint: nivel\_confidentialitate $\in$ \{1, 2, 3\}
    \end{itemize}
    
    \item[\textbf{ENCRYPTION\_KEYS}]
    \begin{itemize}
        \item Tabel auxiliar pentru stocarea cheilor de criptare AES-256
    \end{itemize}
\end{description}

\section{Crearea tabelelor}

Scriptul de creare a tabelelor este disponibil în fișierul.

\begin{figure}[h]
    \centering
    \includegraphics[width=0.5\textwidth]{images/assets/Schema.png}
    \caption{Schema CareConnect}
    \label{fig:schema}
\end{figure}

Structura cheilor primare și a relațiilor este următoarea:

\begin{itemize}
    \item \textbf{Chei primare}: id\_departament, id\_personal, id\_pacient, id\_fisa, key\_id
    \item \textbf{Secvențe}: seq\_departament, seq\_personal, seq\_pacient, seq\_fisa
    \item \textbf{Relații}: 
    \begin{itemize}
        \item PERSONAL\_MEDICAL.id\_departament $\rightarrow$ DEPARTAMENT.id\_departament
        \item FISA\_MEDICALA.id\_pacient $\rightarrow$ PACIENT.id\_pacient
        \item FISA\_MEDICALA.id\_medic $\rightarrow$ PERSONAL\_MEDICAL.id\_personal
    \end{itemize}
\end{itemize}

\section{Reguli de securitate}

Fiecare personal\_medical are propriul user in Database creat cu privilegiile acordate ulterior pe baza gradului/rolului de acces dat la creare; pentru demo avem 4 înregistrari, 1 pentru fiecare rol și mai jos se poate observa existența lor ca useri in baza de date; adăugarea unor noi membrii în personalul medical creează un alt user respectiv in baza de date (se poate observa și în audit log).

\begin{figure}[h]
    \centering
    \includegraphics[width=0.9\textwidth]{images/assets/Screenshot 2026-01-22 at 23.47.31.png}
    \caption{DB Users}
    \label{fig:DB Users}
\end{figure}

\begin{figure}[h]
    \centering
    \includegraphics[width=0.9\textwidth]{images/assets/Screenshot 2026-01-22 at 23.48.51.png}
    \caption{DB Users Audit}
    \label{fig:DB Users audit}
\end{figure}

\subsection{Controlul accesului}

\begin{enumerate}
    \item \textbf{Role-Based Access Control (RBAC)}: Sistemul definește patru roluri ierarhice:
    \begin{itemize}
        \item \texttt{ROL\_RECEPTIE} (grad\_acces=1): poate înregistra pacienți noi, citește date limitate
        \item \texttt{ROL\_ASISTENT} (grad\_acces=2): moștenește privilegiile recepției, poate citi fișe medicale cu nivel $\leq$ 2
        \item \texttt{ROL\_MEDIC} (grad\_acces=3): moștenește privilegiile asistentului, poate crea/modifica fișe medicale, poate decripta CNP
        \item \texttt{ROL\_ADMIN} (grad\_acces=4): acces complet, poate gestiona personalul, roti chei de criptare, accesează audit logs
    \end{itemize}
    
    \item \textbf{Virtual Private Database (VPD)}: Implementat prin Row-Level Security (RLS) pentru filtrarea automată a fișelor medicale pe baza nivelului de confidențialitate și gradului de acces al utilizatorului
    
    \item \textbf{Views cu mascare}: Fiecare rol are view-uri dedicate care maschează datele sensibile (CNP, email, telefon, adresă) conform nivelului de acces
\end{enumerate}

Mai jos se pot oberva exemple de acces diferite bazate pe fiecare rol/grad de acces:

\begin{figure}[h]
    \centering
    \includegraphics[width=0.9\textwidth]{images/assets/RBAC_Proj/Screenshot 2026-01-22 at 23.05.48.png}
    \caption{Fisa medical - acces grad 1}
    \label{fig:rbac}
\end{figure}

\begin{figure}[h]
    \centering
    \includegraphics[width=0.9\textwidth]{images/assets/RBAC_Proj/Screenshot 2026-01-22 at 23.07.33.png}
    \caption{Fisa medical - acces grad 2}
    \label{fig:rbac}
\end{figure}

\begin{figure}[h]
    \centering
    \includegraphics[width=0.9\textwidth]{images/assets/RBAC_Proj/Screenshot 2026-01-22 at 23.07.59.png}
    \caption{Fisa medical - acces grad 3}
    \label{fig:rbac}
\end{figure}

\begin{figure}[h]
    \centering
    \includegraphics[width=0.9\textwidth]{images/assets/RBAC_Proj/Screenshot 2026-01-22 at 23.09.26.png}
    \caption{Fisa medical - acces grad 0}
    \label{fig:rbac}
\end{figure}

\subsection{Criptarea datelor}

\begin{enumerate}
    \item \textbf{Criptare CNP}: CNP-ul pacienților este criptat folosind DBMS\_CRYPTO cu algoritmul AES-256 în modul CBC
\begin{figure}[h]
    \centering
    \includegraphics[width=0.9\textwidth]{images/assets/crypt/Screenshot 2026-01-22 at 23.14.45.png}
    \caption{CNP Criptat}
    \label{fig:crypt cnp}
\end{figure}
    \item \textbf{Stocare chei}: Cheile de criptare sunt stocate într-un tabel separat (ENCRYPTION\_KEYS) cu acces restricționat
\begin{figure}[h]
    \centering
    \includegraphics[width=0.9\textwidth]{images/assets/crypt/Screenshot 2026-01-22 at 23.16.39.png}
    \caption{Chei de criptare AES256-CBC}
    \label{fig:AES256 CBC}
\end{figure}
    \item \textbf{Rotire chei}: Procedură implementată pentru rotirea periodică a cheilor de criptare
\begin{figure}[h]
    \centering
    \includegraphics[width=0.9\textwidth]{images/assets/crypt/Screenshot 2026-01-22 at 23.22.32.png}
    \caption{Rotire chei de criptare}
    \label{fig:rotire chei}
\end{figure}
\begin{figure}[h]
    \centering
    \includegraphics[width=0.9\textwidth]{images/assets/crypt/Screenshot 2026-01-22 at 23.23.43.png}
    \caption{Lista chei}
    \label{fig:lista chei}
\end{figure}
\begin{figure}[h]
    \centering
    \includegraphics[width=0.9\textwidth]{images/assets/crypt/Screenshot 2026-01-22 at 23.26.16.png}
    \caption{Audit Rotire chei}
    \label{fig:audit rotire chei}
\end{figure}
    \item \textbf{Audit decriptări}: Toate decriptările CNP-ului sunt înregistrate în audit log

\end{enumerate}

\begin{figure}[h]
    \centering
    \includegraphics[width=0.9\textwidth]{images/assets/crypt/Screenshot 2026-01-22 at 23.27.56.png}
    \caption{Audit decriptare CNP}
    \label{fig:audit decriptare CNP}
\end{figure}

\subsection{Auditarea}

\begin{enumerate}
    \item \textbf{Trigger-i de auditare}: Implementați pentru înregistrarea modificărilor (INSERT/UPDATE/DELETE) pe tabelele principale cu capturarea valorilor vechi și noi
\begin{figure}[h]
    \centering
    \includegraphics[width=0.9\textwidth]{images/assets/audit_proj/Screenshot 2026-01-22 at 23.31.21.png}
    \caption{Audit prin triggers}
    \label{fig:trigger audits}
\end{figure}
    \item \textbf{Fine-Grained Auditing (FGA)}: Politici configurate pentru auditarea accesărilor la coloane sensibile (CNP, chei de criptare, fișe cu nivel confidențialitate 3)
\begin{figure}[h]
    \centering
    \includegraphics[width=0.9\textwidth]{images/assets/audit_proj/Screenshot 2026-01-22 at 23.33.32.png}
    \caption{Audit FGA}
    \label{fig:trigger FGA}
\end{figure}
\end{enumerate}

In aplicația CLI, doar cine are gradul 4 poate avea acces la audit logs:

\begin{figure}[h]
    \centering
    \includegraphics[width=0.9\textwidth]{images/assets/audit_proj/Screenshot 2026-01-22 at 23.34.47.png}
    \caption{Audit - grad 4}
    \label{fig:audit granted}
\end{figure}

\begin{figure}[h]
    \centering
    \includegraphics[width=0.9\textwidth]{images/assets/audit_proj/Screenshot 2026-01-22 at 23.35.08.png}
    \caption{Audit - grad < 4}
    \label{fig:audit interzis}
\end{figure}

\subsection{Prevenirea atacurilor}

\begin{enumerate}
    \item \textbf{SQL Injection}: Toate operațiile sunt realizate prin proceduri și funcții PL/SQL cu parametri bind, eliminând construirea dinamică de query-uri
    \item \textbf{Application Context}: Context Oracle pentru stocarea gradului de acces în sesiune, setat automat la login
    \item \textbf{Profiluri utilizatori}: Configurate pentru limitarea resurselor (CPU, sesiuni, timp idle) și gestionarea parolelor
\end{enumerate}

\subsection{Mascarea datelor}

\begin{enumerate}
    \item \textbf{Funcții de mascare}: Implementate pentru telefon, email, adresă și CNP (mascare parțială sau totală)
    \item \textbf{Views per rol}: Fiecare rol are view-uri dedicate care aplică mascarea corespunzătoare nivelului de acces
\begin{figure}[h]
    \centering
    \includegraphics[width=0.9\textwidth]{images/assets/mascare/Screenshot 2026-01-22 at 23.36.50.png}
    \caption{View pacienti - grad 1}
    \label{fig:view grad 1}
\end{figure}
\begin{figure}[h]
    \centering
    \includegraphics[width=0.9\textwidth]{images/assets/mascare/Screenshot 2026-01-22 at 23.38.09.png}
    \caption{View pacienti - grad 3}
    \label{fig:view grad 3}
\end{figure}
    \item \textbf{Decriptare condiționată}: Doar medicii și adminii pot decripta CNP-ul complet.
\end{enumerate}

\begin{figure}[h]
    \centering
    \includegraphics[width=0.9\textwidth]{images/assets/crypt/Screenshot 2026-01-22 at 23.41.23.png}
    \caption{Decriptare CNP}
    \label{fig:Decriptare CNP}
\end{figure}
